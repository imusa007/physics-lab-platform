\documentclass[12pt]{article}

\usepackage[margin=1in]{geometry}
\usepackage{amsmath,amssymb}
\usepackage{graphicx}
\usepackage{booktabs}
\usepackage{array}
\usepackage{siunitx}
\usepackage{enumitem}

\sisetup{per-mode=symbol}

\setlength{\parskip}{0.8ex}
\setlength{\parindent}{0pt}

\begin{document}

\title{Lab -- Module 4 -- PHYS 131\\[0.5em]Two-Dimensional Motion}
\date{}
\maketitle

\textbf{Student Name:} {{ student_name }}\\
\textbf{Date:} {{ date }}\\
\textbf{Lab Section:} {{ section }}\\[1em]

\section*{Introduction}

It is rare for objects to move in only a single direction in nature. More often they
follow a curved path like the flight of a football or a planet in orbit. We usually call
this type of motion \emph{projectile motion}.

During projectile motion, objects are moving in both the horizontal ($x$) and vertical
($y$) directions at the same time. Luckily, we can examine motion in either direction
separately. The 1D kinematics equations still apply to both the $x$ coordinate and the
$y$ coordinate individually.

\subsubsection*{General 2D kinematics}

\begin{minipage}{0.48\textwidth}
\textbf{Horizontal Motion}
\begin{align*}
x_f &= x_i + v_{ix} t + \frac{1}{2} a_x t^2 \\
v_{fx} &= v_{ix} + a_x t \\
v_{fx}^2 &= v_{ix}^2 + 2 a_x (x_f - x_i)
\end{align*}
\end{minipage}
\hfill
\begin{minipage}{0.48\textwidth}
\textbf{Vertical Motion}
\begin{align*}
y_f &= y_i + v_{iy} t + \frac{1}{2} a_y t^2 \\
v_{fy} &= v_{iy} + a_y t \\
v_{fy}^2 &= v_{iy}^2 + 2 a_y (y_f - y_i)
\end{align*}
\end{minipage}

These can be simplified further by recognizing that acceleration occurs only in the
vertical ($y$) direction because the only force acting on the object while it is in the
air (neglecting friction and air resistance) is gravity. Gravity has a direction that is
straight toward the center of the Earth, which we define as the negative $y$ direction.
This means that $a_x = 0$ and $a_y = -g$. The magnitude of the acceleration due to
gravity is $g = \SI{9.80}{\meter\per\second\squared}$.

\subsubsection*{Projectile motion with gravity only}

\begin{minipage}{0.48\textwidth}
\textbf{Horizontal Motion}
\begin{align*}
x_f &= x_i + v_{ix} t \\
v_{fx} &= v_{ix}
\end{align*}
\end{minipage}
\hfill
\begin{minipage}{0.48\textwidth}
\textbf{Vertical Motion}
\begin{align*}
y_f &= y_i + v_{iy} t - \frac{1}{2} g t^2 \\
v_{fy} &= v_{iy} - g t \\
v_{fy}^2 &= v_{iy}^2 - 2 g (y_f - y_i)
\end{align*}
\end{minipage}

The important note is that these equations are linked by time. The value of $t$ is the
same for both equations involving $x$ and equations involving $y$ when we examine the
same point in the flight of the object.

During the flight of an object, the total velocity is always the vector sum of the
horizontal component and vertical component of velocity.

\begin{figure}[h!]
    \centering
    \includegraphics[width=\textwidth]{diagram_page2.png}
    \caption{Horizontal and vertical components of projectile motion.}
\end{figure}

Each position of the ball in the diagram is one second apart. The horizontal component of
velocity is constant (equal spacing), while the vertical component of velocity and
displacement change as the ball falls, indicating acceleration due to gravity.

\section*{Procedure}

\begin{enumerate}[label=\arabic*.]
    \item Connect each piece of your Hot Wheel track to be used as a ramp. Set up the
    ramp on a table or countertop so that the upper end of the ramp is about
    \SI{10}{\centi\meter} above the tabletop. Record the actual ramp height as
    \textbf{Ramp Height 1} in the Data Section.

    \item Adjust the position of your ramp so the lower end of the ramp is at least
    \SI{1}{\meter} away from the edge of the table/counter. Timing the ball becomes
    easier the longer this distance is. Measure the actual distance between the bottom
    of the ramp and the end of the table/counter. Record this distance in Table~1 in
    the Data Section.

    \item Release the ball from the upper end of the ramp and time how long it takes
    for the ball to travel from when it leaves the bottom of the ramp to when it
    leaves the edge of the table. Only time the horizontal part of the ball's motion
    along the table; ignore the time it spends on the ramp. Repeat this measurement
    two more times and record your times in the remaining two rows of the data table.

    \item Calculate the speed of the ball when it leaves the table for each of your
    three trials using your measured time and horizontal distance along the table.
    This will be your horizontal velocity, $v_{ix}$, at the beginning of projectile
    motion.

    \item Measure the vertical distance from the floor up to the table/countertop
    (where the ball leaves the table/counter and becomes airborne). Calculate the time
    you expect the ball to be in the air for all three trials.

    \item Calculate the horizontal distance you expect the ball to travel while in the
    air for each trial. This means the horizontal distance away from the edge of the
    table that the ball will hit the ground.

    \item Find the average for each of your calculated values.

    \item Place a cup at your calculated landing spot and see if you can get the ball
    to land in the cup. You are not allowed to move it closer to or farther from the
    table. It must be \emph{at} your calculated position. Find the calculated landing
    spot by drawing an imaginary line straight down to the ground from the edge of the
    table, then use your measuring tape to measure to your calculated horizontal
    position. Record whether the ball lands in front of the cup, at/in the cup, or
    beyond the cup below the data table.

    \item Repeat steps 1 through 8 at two more ramp heights. There are additional data
    tables in the Data Section for the new ramp heights.
\end{enumerate}

\section*{Data}

%========================
% RAMP 1
%========================

\subsection*{Ramp Height 1: {{ ramp1.height }}}

\begin{center}
\textbf{Table 1}
\end{center}

\begin{center}
\begin{tabular}{cccccc}
\toprule
\begin{tabular}{c}Distance from\\end of ramp to\\table edge (m)\end{tabular} &
\begin{tabular}{c}Time for ball to\\travel along\\horizontal tabletop (s)\end{tabular} &
\begin{tabular}{c}Velocity when\\ball leaves table\\(m/s) $v_{ix}$\end{tabular} &
\begin{tabular}{c}Height of\\table top (m)\\$y_i$\end{tabular} &
\begin{tabular}{c}Calculated time\\ball is in the air (s)\\$t$\end{tabular} &
\begin{tabular}{c}Calculated horizontal\\distance travelled\\in air (m) $x_f$\end{tabular} \\
\midrule

{{ row.distance }} & {{ row.time }} & {{ row.vix }} & {{ row.yi }} & {{ row.t }} & {{ row.xf }} \\

\midrule
\textbf{Averages:} &
{{ ramp1.avg.distance }} &
{{ ramp1.avg.time }} &
{{ ramp1.avg.vix }} &
{{ ramp1.avg.t }} &
{{ ramp1.avg.xf }} \\
\bottomrule
\end{tabular}
\end{center}

\vspace{1ex}

Where did the ball land?\\[0.5ex]

\textbf{Before the cup}

\textbf{At or in the cup}

\textbf{Past the cup}

\textit{(Not recorded)}


\vspace{1.5ex}

\textbf{Example calculation using average values:}\\
{{ ramp1.example_calculation }}

%========================
% RAMP 2
%========================

\subsection*{Ramp Height 2: {{ ramp2.height }}}

\begin{center}
\textbf{Table 2}
\end{center}

\begin{center}
\begin{tabular}{cccccc}
\toprule
\begin{tabular}{c}Distance from\\end of ramp to\\table edge (m)\end{tabular} &
\begin{tabular}{c}Time for ball to\\travel along\\horizontal tabletop (s)\end{tabular} &
\begin{tabular}{c}Velocity when\\ball leaves table\\(m/s) $v_{ix}$\end{tabular} &
\begin{tabular}{c}Height of\\table top (m)\\$y_i$\end{tabular} &
\begin{tabular}{c}Calculated time\\ball is in the air (s)\\$t$\end{tabular} &
\begin{tabular}{c}Calculated horizontal\\distance travelled\\in air (m) $x_f$\end{tabular} \\
\midrule

{{ row.distance }} & {{ row.time }} & {{ row.vix }} & {{ row.yi }} & {{ row.t }} & {{ row.xf }} \\

\midrule
\textbf{Averages:} &
{{ ramp2.avg.distance }} &
{{ ramp2.avg.time }} &
{{ ramp2.avg.vix }} &
{{ ramp2.avg.t }} &
{{ ramp2.avg.xf }} \\
\bottomrule
\end{tabular}
\end{center}

\vspace{1ex}

Where did the ball land?\\[0.5ex]

\textbf{Before the cup}

\textbf{At or in the cup}

\textbf{Past the cup}

\textit{(Not recorded)}


%========================
% RAMP 3
%========================

\subsection*{Ramp Height 3: {{ ramp3.height }}}

\begin{center}
\textbf{Table 3}
\end{center}

\begin{center}
\begin{tabular}{cccccc}
\toprule
\begin{tabular}{c}Distance from\\end of ramp to\\table edge (m)\end{tabular} &
\begin{tabular}{c}Time for ball to\\travel along\\horizontal tabletop (s)\end{tabular} &
\begin{tabular}{c}Velocity when\\ball leaves table\\(m/s) $v_{ix}$\end{tabular} &
\begin{tabular}{c}Height of\\table top (m)\\$y_i$\end{tabular} &
\begin{tabular}{c}Calculated time\\ball is in the air (s)\\$t$\end{tabular} &
\begin{tabular}{c}Calculated horizontal\\distance travelled\\in air (m) $x_f$\end{tabular} \\
\midrule

{{ row.distance }} & {{ row.time }} & {{ row.vix }} & {{ row.yi }} & {{ row.t }} & {{ row.xf }} \\

\midrule
\textbf{Averages:} &
{{ ramp3.avg.distance }} &
{{ ramp3.avg.time }} &
{{ ramp3.avg.vix }} &
{{ ramp3.avg.t }} &
{{ ramp3.avg.xf }} \\
\bottomrule
\end{tabular}
\end{center}

\vspace{1ex}

Where did the ball land?\\[0.5ex]

\textbf{Before the cup}

\textbf{At or in the cup}

\textbf{Past the cup}

\textit{(Not recorded)}


\section*{Questions}

\begin{enumerate}[label=\arabic*.]
    \item What two values affect the time the ball is in the air? (Hint: You measured
    one of them during the lab and the other is a universal constant.)\\[0.3em]
    \textbf{Answer:} {{ q1 }}

    \item Is the time the ball spends in the air affected by how fast it is moving when
    it leaves the table? Why or why not?\\[0.3em]
    \textbf{Answer:} {{ q2 }}

    \item Did the ball land in the position you expected at any of the ramp heights
    (did it ever land in the cup)? Was the actual place the ball landed closer or
    farther from the table than the calculated value? Why do you think this is?\\[0.3em]
    \textbf{Answer:} {{ q3 }}

    \item What are some sources of error in your experimental measurements that would
    cause you to calculate a longer distance than what is actually observed?
    Experimental errors are things that we have ignored during calculations or things
    that affect our measurements, but we cannot control.\\[0.3em]
    \textbf{Answer:} {{ q4 }}

    \item Now consider doing the same experiment, except you place a second, upward
    ramp at the edge of the table so that the ball is moving in both the $x$ and the
    $y$ directions when it leaves the table. The ramp causes the ball to leave the
    table at \SI{2.10}{\meter\per\second} at an angle of $30^\circ$ above the
    horizontal.
    \begin{enumerate}[label=\alph*)]
        \item {{ q5a }}
        \item {{ q5b }}
        \item {{ q5c }}
        \item {{ q5d }}
        \item {{ q5e }}
    \end{enumerate}
\end{enumerate}

\end{document}
